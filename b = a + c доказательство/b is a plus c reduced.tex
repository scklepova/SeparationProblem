\documentclass[12pt]{article}
\usepackage[utf8]{inputenc}
\usepackage{amsthm}
\usepackage{amsmath}
\usepackage{amsfonts}
\usepackage{mathtools}
\usepackage{graphicx}
\usepackage{mathrsfs}
\usepackage{verbatim}

\usepackage[utf8]{inputenc}
\usepackage[english,russian]{babel}
\usepackage{geometry}
%\usepackage{textcase}
\usepackage{multicol}
\usepackage{listings}
\usepackage{extsizes}
\usepackage{indentfirst}
\usepackage{graphicx}
\usepackage{float}
%\usepackage{algorithmicx}
\usepackage{algorithm}
%\usepackage{algpseudocode}
\usepackage{amsthm}

\newtheorem{lemma}{Лемма}
\newtheorem{definition}{Определение}
\newtheorem{theorem}{Теорема}
\newtheorem{corollary}{Следствие}

\DeclareMathOperator{\lcm}{lcm}
\usepackage{extsizes}
\graphicspath{ {images/} }


\newcommand{\image}[2]{
	\begin{figure}[H]
		\center{
			\includegraphics{#1}
		}
		\caption{#2}
	\end{figure}
}

\newcommand{\scalingimage}[3]{
	\begin{figure}
		\center{
			\includegraphics[scale=#2]{#1}
		}
		\caption{#3}
		\label{fig:#1}
	\end{figure}
}

\begin{document}
		$(xy)^a(yx)^b(xy)^c$ = $(yx)^c(xy)^b(yx)^a$ является тождеством в $S_k$, если для каждого $z$ - порядка $k$-перестановки выполняется хотя бы одно из следующих правил:
		\begin{equation} \label{3blocks 1st cond}
		z|a \hspace{10pt} \text{и} \hspace{10pt} z|(b-c)
		\end{equation}
		\begin{equation}
		z|c \hspace{10pt} \text{и} \hspace{10pt} z|(b-a)
		\end{equation}
		\begin{equation} \label{3blocks 3rd cond}
		z|b \hspace{10pt} \text{и} \hspace{10pt} z|(a+c)
		\end{equation}
	
	\begin{definition}
		Циклической перестановкой из $k$ элементов с шагом $s$ будем называть такую циклическую перестановку, в которой элемент с номером $i$ переходит в элемент с номером $i+s \hspace{5pt} (\mod k)$. 
	\end{definition}
	
	Далее будем считать, что элементы перестановки длины $k$ - это числа от $0$ до $k-1$.
	
	\begin{comment}
	\begin{lemma} \label{existence_of_clockwise_and_vv}
		Существуют такие перестановки $x$ и $y$ из $S_k$, что $xy$ - циклическая перестановка с шагом -1, а $yx$ - циклическая перестановка с шагом 1.
	\end{lemma}
	\begin{proof}
		Рассмотрим перестановку $x$: $i \rightarrow -i (\mod k)$ и $y$: $i \rightarrow -i+1 (\mod k)$.
		$$
		x = 
		\begin{pmatrix}
		0&1&2&3&...&k-3&k-2&k-1\\
		0&k-1&k-2&k-3&...&3&2&1
		\end{pmatrix}
		$$
		
		$$
		y = 
		\begin{pmatrix}
		0&1&2&3&...&k-3&k-2&k-1\\
		1&0&k-1&k-2&...&4&3&2
		\end{pmatrix}
		$$
		
		Тогда 
		
		$xy$: $i \xrightarrow y (-i+1) \xrightarrow x (-(-i+1)) == i-1$,
		
		$yx$: $i \xrightarrow x (-i) \xrightarrow y (-(-i)+1) == i+1$
		
	\end{proof}
	\end{comment}
	
	\begin{lemma} \label{a-b+c is 0 for odd}
		Пусть $(xy)^a(yx)^b(xy)^c$ = $(yx)^c(xy)^b(yx)^a$ - тождество в $S_k$, где $k$ - нечетное. Тогда $a - b + c \equiv 0 \hspace{5pt} (\mod k)$.
	\end{lemma}
	\begin{proof}
		Зафиксируем перестановки $xy$ и $yx$ %из Леммы \ref{existence_of_clockwise_and_vv}.
		такие, что $xy$ - циклическая перестановка с шагом -1, а $yx$ - циклическая перестановка с шагом 1.
		Рассмотрим перестановочный автомат, в котором переход по символам осуществляется соответствующими перестановками $x$ и $y$. Тогда, чтобы $(xy)^a(yx)^b(xy)^c$ = $(yx)^c(xy)^b(yx)^a$ было тождеством для такого автомата, требуется, чтобы автомат закончил читать обе части равенства в одном состоянии, то есть	
		\begin{equation}
		(-a) + b + (-c) \equiv c + (-b) + a \hspace{5pt} (\mod k)
		\end{equation}
		что эквивалентно 
		\begin{equation}
		2(a - b + c) \equiv 0 \hspace{5pt} (\mod k)
		\end{equation}
		Из того, что $k$ нечетно, следует 
		$$
		(a - b + c) \equiv 0 \hspace{5pt} (\mod k)
		$$.
	\end{proof}

	\begin{lemma} \label{a-b+c is 0 for even}
		Пусть $(xy)^a(yx)^b(xy)^c$ = $(yx)^c(xy)^b(yx)^a$ - тождество в $S_k$, где $k$ - четное. Тогда $a - b + c \equiv 0 \hspace{5pt} (\mod \frac{k}{2})$.
	\end{lemma}
	\begin{proof}
		Рассмотрим перестановочный автомат, как в доказательстве Леммы \ref{a-b+c is 0 for odd}. Аналогично, получим	
		\begin{equation}
		2(a - b + c) \equiv 0 \hspace{5pt} (\mod k)
		\end{equation}
		Из того, что $k$ четно, следует 
		$$
		(a - b + c) \equiv 0 \hspace{5pt} (\mod \frac{k}{2})
		$$.
	\end{proof}

	\begin{corollary} \label{a-b+c is 0 mod lcm/2}
		Пусть $(xy)^a(yx)^b(xy)^c$ = $(yx)^c(xy)^b(yx)^a$ - тождество в $S_k$. Тогда $a - b + c \equiv 0 \hspace{5pt} (\mod \frac{\lcm(k)}{2})$.
	\end{corollary}
	\begin{proof}
		Из Леммы \ref{a-b+c is 0 for odd} $a - b + c \equiv 0$ по модулю наименьшего общего кратного всех нечетных чисел, меньших $k$.
		По Лемме \ref{a-b+c is 0 for even} $a - b + c \equiv 0$ по модулю предмаксимальной степени числа 2, не превосходящей $k$.
		Отсюда следует, что $a - b + c \equiv 0 \hspace{5pt} (\mod \frac{\lcm(k)}{2})$.
	\end{proof}

	\begin{theorem}\label{b is a plus c}
		Пусть $(xy)^a(yx)^b(xy)^c$ = $(yx)^c(xy)^b(yx)^a$ - тождество в $S_k$ и $a + b + c \le \frac{\lcm(k)}{2}$. Тогда $b = a+c$.
	\end{theorem}
	\begin{proof}
		Сразу заметим, что ни одно из чисел $a$, $b$, $c$ не равно 0, так как в противном случае мы будем рассматривать тождество другого типа.
		
		Из Следствия \ref{a-b+c is 0 mod lcm/2} вытекает, что \hspace{5pt} 
		$$
			b - a - c = \frac{\lcm(k)}{2}m,
		$$ где $m \in \mathbb{Z}$. 
		Откуда
		$$
			b = a + c + \frac{\lcm(k)}{2}m > 0.
		$$
		Получим цепочку неравенств 
		$$
			0 < a + c + \frac{\lcm(k)}{2}m < \frac{\lcm(k)}{2} + \frac{\lcm(k)}{2}m = \frac{\lcm(k)}{2}(m + 1),
		$$то есть $m > -1$.
		
		С другой стороны, подставим $b$ в $a+b+c$, получим
		$$
			2(a + c) + \frac{\lcm(k)}{2}m \le \frac{\lcm(k)}{2}
		$$ или 
		$$
			2(a + c) \le \frac{\lcm(k)}{2}(1-m)
		$$
		Поскольку сумма $a$ и $c$ должна быть положительным числом, требуется
		$$
			\frac{\lcm(k)}{2}(1-m) > 0,
		$$ то есть $m < 1$.
		
		Значит, при заданных ограничениях $m = 0$, что влечет $b = a + c$.
	\end{proof}

	\begin{corollary}
		Для кратчайшего тождества вида $(xy)^a(yx)^b(xy)^c$ = $(yx)^c(xy)^b(yx)^a$ выполняется $b = a + c$.
	\end{corollary}
	\begin{proof}
		Достаточно показать, что существуют тождества, для которых $a + b + c \le \frac{\lcm(k)}{4}$ (тогда кратчайшее тождество также удовлетворяет этому условию, а по теореме для всех тождеств с таким свойством выполняется $b = a + c$).
		
		Пусть $m = 2k/3$, $a := lcm(m)$, $c := lcm(k - m) \cdot P(m)$, $b := a+c$, где $P(m)$ - произведение всех простых и степеней простых чисел из множества $\{m+1, ..., k\}$. $a$ и $c$ взяты из доказательства длины тождества из двух блоков. Оттуда же понятно, что любой порядок перестановки делит или $a$, или $c$, а, благодаря выбору $b$, делит и соответствующую разность. Длина такого тождества, конечно, в два раза больше длины тождества из двух блоков ($e^{\frac{2}{3}k + O(\frac{k}{\log k})}$), однако все равно асимптотически меньше, чем $\frac{\lcm(k)}{4}$ (который  равен $e^{k + O(\frac{k}{\log k})}$).
	\end{proof}

	Теперь, вооружившись утверждением о связи показателей степеней рассматриваемого тождества, можно доказать обратное утверждение, т.е. если $(xy)^a(yx)^b(xy)^c$ $\equiv_k$ $(yx)^c(xy)^b(yx)^a$, то выполняется хотя бы одно из условий \ref{3blocks 1st cond} - \ref{3blocks 3rd cond}. Для этого нам понадобится доказать еще несколько лемм.
	Однако заметим сразу, что равенство $b = a + c$ делает истинной вторые части утверждений \ref{3blocks 1st cond} - \ref{3blocks 3rd cond}, если первые истинны.
	
	\begin{lemma}\label{abc div 2}
		 Если $(xy)^a(yx)^b(xy)^c$ $\equiv_k$ $(yx)^c(xy)^b(yx)^a$, тогда хотя бы одно из чисел $a$, $b$, $c$ делится на 2.
	\end{lemma}
	\begin{proof}
		От противного. Пусть $a$, $b$, $c$ нечетны. Тогда $b \ne a+c$, поскольку их четность не совпадает. Противоречие.
	\end{proof}

	\begin{lemma}\label{abc div 3}
		Если $(xy)^a(yx)^b(xy)^c$ $\equiv_k$ $(yx)^c(xy)^b(yx)^a$ и $k \ge 4$, тогда хотя бы одно из чисел $a$, $b$, $c$ делится на 3.
	\end{lemma}
	\begin{proof}
		От противного. Пусть ни одно из чисел $a$, $b$, $c$ не делится на 3. Тогда возможны лишь два варианта:
		\begin{enumerate}
			\item $a \equiv_3 c \equiv_3 1$ и $b \equiv_3 2$
			\item $a \equiv_3 c \equiv_3 2$ и $b \equiv_3 1$
		\end{enumerate}
	
		(В остальных случаях хотя бы одно из чисел оказывается кратным трем)
		
		Рассмотрим автомат относительно символов $xy$, $yx$ на рисунке \ref{fig:3division}. Такой автомат можно получить, взяв за перестановку по $x$ $(0)(1,2,3)$, по $y$ - $(1)(3, 2, 0)$. Начальное состояние 0.
		
		В первом случае автомат закончит читать левую часть тождества в состоянии 3 (после прочтения $(xy)^a$ окажется в состоянии 1, затем, прочитав $(yx)^b$ придет в состояние 3 в цикле), а правую - в состоянии 1 (после прочтения $(yx)^c$ окажется в состоянии 3 и в нем останется после $(xy)^b$).
		
		\scalingimage{3division}{0.5}{}
		
		Во втором случае автомат закончит читать левую часть тождества в состоянии 1 (после прочтения $(xy)^a$ окажется в состоянии 2, затем, прочитав $(yx)^b$ останется в состоянии 2), а правую - в состоянии 2 (после прочтения $(yx)^c$ окажется в состоянии 1 и перейдет в состояние 2 после $(xy)^b$).
		
		Получили противоречие с тем, что данная пара - тождество.
	\end{proof}

	\begin{lemma}\label{abc div k}
		Если $(xy)^a(yx)^b(xy)^c$ $\equiv_k$ $(yx)^c(xy)^b(yx)^a$ и $k \ge 4$, тогда хотя бы одно из чисел $a$, $b$, $c$ делится на $k$.
	\end{lemma}
	\begin{proof}
		От противного. Пусть ни одно из чисел $a$, $b$, $c$ не делится на $k$. Разберем несколько случаев.
		\begin{enumerate}
			\item $2a \equiv_k 0$.
			\scalingimage{cycle_a_c}{0.5}{Цикл $xy$ автомата $\mathscr{A}$}
			
			Поскольку мы предположили, что $a \not \equiv_k 0$, значит $a \equiv_k \frac{k}{2}$. Так как $a+c=b$, $b \not \equiv_k 0$ по предположению, то $c \not \equiv_k a$.
			
			\scalingimage{process1_a_c}{0.5}{Чтение слова $(xy)^a(yx)^b(xy)^c$ автоматом $\mathscr{A}$}
			\scalingimage{process2_a_c}{0.5}{Чтение слова $(yx)^c(xy)^b(yx)^a$ автоматом $\mathscr{A}$}
			
			Рассмотрим автомат $\mathscr{A}$, в котором $xy$ - перестановка с шагом 1, в которой поменяли местами $a$ и $c$, а $yx$ - перестановка с шагом -1 (см. Рисунок \ref{fig:cycle_a_c}). Данный автомат различит слова $(xy)^a(yx)^b(xy)^c$ и $(yx)^c(xy)^b(yx)^a$. 
						
			$\mathscr{A}$ закончит читать второе слово в состоянии $c-a$ (см. Рисунок \ref{fig:process2_a_c}).
			Корректность переходов:
			\begin{itemize}
				\item $-c \not \equiv_k a$ и $-c \not \equiv_k c$ поскольку иначе в обоих случаях он был бы равен $\frac{k}{2}$ и совпадал бы с $a$;
				\item $b-c = a$ из Теоремы \ref{b is a plus c}.
			\end{itemize}
			
			$\mathscr{A}$ закончит читать второе слово в состоянии с номером $2c$ (см. Рисунок \ref{fig:process2_a_c}).
			Корректность переходов:
			\begin{itemize}
				\item $c-b \equiv_k -a$ из Теоремы \ref{b is a plus c}, а $a \equiv_k -a$ поскольку $a \equiv_k \frac{k}{2}$;
				\item $2c \not \equiv_k c$ поскольку по предположению $c \not \equiv_k 0$.
			\end{itemize}
			Конечное состояние при чтении $(yx)^c(xy)^b(yx)^a$ зависит от того, совпадает ли $2c$ с $a$ по модулю $k$:
			\begin{itemize}
				\item $2c \equiv_k a$, тогда состоянием с номером $2c$ будет состояние $c$, которое не совпадает с $c-a$, поскольку $-a$ не сравнимо с нулем по модулю $k$;
				\item $2c \not \equiv_k a$, тогда конечным состоянием будет $2c$, которое не совпадает с $c-a$, поскольку $a+c \not \equiv_k 0$.
			\end{itemize}
			
			\item $2a \not \equiv_k 0$ и $2a \not \equiv_k -c$
			\scalingimage{cycle_a_2a}{0.5}{Цикл $yx$ автомата $\mathscr{B}$}
			\scalingimage{process1_a_2a}{0.5}{Чтение слова $(xy)^a(yx)^b(xy)^c$ автоматом $\mathscr{B}$}
			\scalingimage{process2_a_2a}{0.5}{Чтение слова $(yx)^c(xy)^b(yx)^a$ автоматом $\mathscr{B}$}
			
			Рассмотрим автомат $\mathscr{B}$, в котором $xy$ - перестановка с шагом 1, а $yx$ - перестановка с шагом -1, в которой поменяли местами $a$ и $2a$ (см. Рисунок \ref{fig:cycle_a_2a}). Данный автомат различит слова $(xy)^a(yx)^b(xy)^c$ и $(yx)^c(xy)^b(yx)^a$.
			
			$\mathscr{B}$ закончит читать $(xy)^a(yx)^b(xy)^c$ в состоянии $a$ (см. Рисунок \ref{fig:process1_a_2a}).
			Корректность переходов:
			\begin{itemize}
				\item $2a-b = a-c$ по Теореме \ref{b is a plus c}, $a-c \not \equiv_k a$, так как $-c \not \equiv_k 0$ по предположению, и $a-c \not \equiv_k 2a$, так как $a+c \not \equiv_k 0$ также по предположению.
			\end{itemize}
		
			$\mathscr{B}$ закончит читать $(yx)^c(xy)^b(yx)^a$ в состоянии $2a$ (см. Рисунок \ref{fig:process2_a_2a}).
			Корректность переходов:
			\begin{itemize}
				\item $-c \not \equiv_k a$, поскольку $a+c \not \equiv_k 0$ по предположению, и $-c \not \equiv_k 2a$ по заданному ограничению;
				\item $b-a = c$ по Теореме \ref{b is a plus c}.
			\end{itemize}
			Состояния $a$ и $2a$ не совпадают, так как $a \not \equiv_k 0$ по предположению.
			
			\item $2a \not \equiv_k 0$ и $2a \equiv_k -c$ и $3a \not \equiv_k 0$
			\scalingimage{process2_a_2a_-c_is_2a}{0.5}{Чтение слова $(yx)^c(xy)^b(yx)^a$ автоматом $\mathscr{B}$ в случае, когда $2a \equiv_k -c$}
			
			Рассмотрим автомат $\mathscr{B}$ из предыдущего случая (см. Рисунок \ref{fig:cycle_a_2a}). В этом случае этот автомат также разделит рассматриваемую пару слов.
			
			Чтение слова $(xy)^a(yx)^b(xy)^c$ будет абсолютно таким же, как и в предыдущем случае.
			Чтение слова $(yx)^c(xy)^b(yx)^a$ закончится в состоянии $-a$ (см. Рисунок \ref{fig:process2_a_2a_-c_is_2a}).
			Корректность переходов:
			\begin{itemize}
				\item $2a \equiv_k -c$ по заданному ограничению;
				\item $a+b \equiv_k 0$ поскольку $a+b = 2a+c \equiv_k 2a - 2a = 0$ по Теореме \ref{b is a plus c} и заданному ограничению;
				\item $-a \not equiv_k a$, так как $2a \not \equiv_k 0$, и $-a \not equiv_k a$, поскольку $3a \not \equiv_k 0$ по заданному ограничению.
			\end{itemize}
			Состояния $a$ и $-a$ также не совпадают.
			
			\item $2a \not \equiv_k 0$ и $2a \equiv_k -c$ и $3a \equiv_k 0$
			
			Поскольку $3a \equiv_k 0$, то $a \equiv_k \frac{k}{3}$ или $a \equiv_k \frac{2k}{3}$, а из того, что $2a+c \equiv_k 0$ следует, что $a \equiv_k c$.
			
			\scalingimage{cycle_2m_z}{0.5}{Цикл $yx$ автомата $\mathscr{C}$}
			
			Пусть $k = 3m$, $m \in \mathbb{N}$.
			Рассмотрим перестановочный автомат $\mathscr{C}$ такой, что $xy$ - циклическая перестановка из $k$ элементов с шагом 1, а $yx$ - циклическая перестановка из $k$ элементов с шагом 1, в которой поменяли местами состояния $2m$ и $z$, где $z \not \equiv 0 (\mod k)$, $z \not \equiv m (\mod k)$ и $z \not \equiv 2m (\mod k)$ (см. Рис. \ref{fig:cycle_2m_z}). Такое $z$ существует, если $k > 3$.
			
			Покажем, что автомат $\mathscr{C}$ закончит читать слова $(xy)^a(yx)^b(xy)^c$ и $(yx)^c(xy)^b(yx)^a$ в разных состояниях.
			
			\begin{enumerate}
				\item $a \equiv_k c \equiv_k m$, $b \equiv_k 2m$
				
				Начальным состоянием в автомате $\mathscr{C}$ назначим состояние $m$.
				\scalingimage{process1_2m_z}{0.5}{Чтение слова $(xy)^a(yx)^b(xy)^c$ автоматом $\mathscr{C}$ при $a \equiv_k \frac{k}{3}$}
				\scalingimage{process2_2m_z}{0.5}{Чтение слова $(yx)^c(xy)^b(yx)^a$ автоматом $\mathscr{C}$ при $a \equiv_k \frac{k}{3}$}
				
				$\mathscr{C}$ закончит читать $(xy)^a(yx)^b(xy)^c$ в состоянии $z$ (см. Рисунок \ref{fig:process1_2m_z}), а слово $(yx)^c(xy)^b(yx)^a$ - в состоянии $2m$ (см. Рисунок \ref{fig:process2_2m_z}).
				
				Корректность переходов:
				\begin{itemize}
					\item $z + 2m \not \equiv_k z$, так как $2m = \equiv_k 2a \not \equiv_k 0$ по заданному ограничению;
					\item $z + 2m \not \equiv_k 2m$, так как $z \not \equiv_k 0$ по выбору $z$.
				\end{itemize}
				Состояние $z$ не совпадает с $2m$ по выбору $z$.
				
				\item $a \equiv_k c \equiv_k 2m$, $b \equiv_k m$
				
				Начальным состоянием в автомате $\mathscr{C}$ назначим состояние $0$.
				\scalingimage{process1_2m_z_init_0}{0.5}{Чтение слова $(xy)^a(yx)^b(xy)^c$ автоматом $\mathscr{C}$ при $a \equiv_k \frac{2k}{3}$}
				\scalingimage{process2_2m_z_init_0}{0.5}{Чтение слова $(yx)^c(xy)^b(yx)^a$ автоматом $\mathscr{C}$ при $a \equiv_k \frac{2k}{3}$}
				
				$\mathscr{C}$ закончит читать $(xy)^a(yx)^b(xy)^c$ в состоянии $z$ (см. Рисунок \ref{fig:process1_2m_z_init_0}), а слово $(yx)^c(xy)^b(yx)^a$ - в состоянии $2m$ (см. Рисунок \ref{fig:process2_2m_z_init_0}).
				
				Корректность переходов:
				\begin{itemize}
					\item $z + m \not \equiv_k z$, так как $m = \equiv_k 2a \not \equiv_k 0$ по заданному ограничению;
					\item $z + m \not \equiv_k m$, так как $z \not \equiv_k 0$ по выбору $z$.
				\end{itemize}
				
				Состояние $z$ не совпадает с $2m$ по выбору $z$.
			\end{enumerate}
		\end{enumerate}
	
		Рассмотрены все возможные случаи, и для каждого из них приведен автомат, разделяющий пару слов, которая по условию была тождеством. Значит, наше предположение о том, что ни одно из чисел $a$, $b$, $c$ не делится на $k$, было ложным.
	\end{proof}

	\begin{theorem}
		Если $(xy)^a(yx)^b(xy)^c$ $\equiv_k$ $(yx)^c(xy)^b(yx)^a$ и $k \ge 4$, то для любого порядка $ord$ перестановки из $k$ элементов хотя бы одно из чисел $a$, $b$, $c$ делится на $ord$.
	\end{theorem}
	\begin{proof}
		Если перестановка представима в виде одного цикла (здесь опускаются тривиальные циклы из одного состояния), значит порядок перестановки $ord$ меньше либо равен $k$, и утверждение доказано Леммами \ref{abc div 2} - \ref{abc div k}.
		
		Пусть k-перестановка представима в виде произведения двух циклов с длинами $c_1$ $c_2$. Тогда порядок перестановки равен $lcm(c_1, c_2)$. От противного, предположим, что ни одно из чисел $a$, $b$, $c$ не делится на $lcm(c_1, c_2)$. Рассмотрим несколько вариантов.
		\scalingimage{2_cycles}{0.5}{Автомат $\mathscr{A}$}
		
		\begin{enumerate}
			\item $c_1 | a$ и $c_2 | b$.
			
			Тогда $c_1 \not | c$ и $c_2 \not | c$, поскольку иначе разность $b-c$ делилась бы на $c_1$ или $c_2$ соответственно, а это повлекло бы за собой или $c_2 | a$, или $c_1 | b$. Также, исходя из Теоремы \ref{b is a plus c}, $c_1 | (b-c)$, откуда $b \equiv_{c_1} c$.
			
			Рассмотрим автомат $\mathscr{A}$ c переходами по $xy$ и $yx$, изображенный на рисунке \ref{fig:2_cycles}. Начальное состояние автомата - $0$; циклы, длинами $c_1$ и $c_2$, у разных перестановок отличаются только одним состоянием (состояние $s$ у перестановки $xy$ располагается в цикле $c_2$, а у перестановки $yx$ - в цикле $c_1$; аналогичная ситуация с состоянием $t$), остальные вершины у циклов с одинаковыми длинами совпадают. Покажем, что данный автомат различит строки $(xy)^a(yx)^b(xy)^c$ и $(yx)^c(xy)^b(yx)^a$.
			
			\begin{itemize}
				\item Читаем $(xy)^a(yx)^b(xy)^c$
				
				$0.(xy)^a = 0$, так как $c_1 | a$. $0.(yx)^b = s$ по выбору вершины $s$, и автомат переходит к циклу $c_2$. $s.(xy)^c = w$, где $w$ - вершина цикла длины $c_1$ перестановки $xy$, не совпадающая с $s$, поскольку $c_1 \not | c$. 
				
				\item Читаем $(yx)^c(xy)^b(yx)^a$
				
				$0.(yx)^c = s$, так как $b \equiv_{c_1} c$; автомат переходит к циклу $c_2$. $s.(xy)^b = s$ поскольку $c_2 | b$;  автомат переходит к циклу $c_1$. $s.(yx)^a = s$, поскольку $c_1 | a$.
			\end{itemize}
			Поскольку $w \ne s$, автомат различит слова, что противоречит тождественности пары.
			
			\item $c_1 | a$ и $c_2 | с$.
			
			Рассуждая аналогично предыдущему случаю, получим, что $b \equiv_{c_1} c$, $c_1 \not | b$, $c_2 \not | b$.
			
			Рассмотрим автомат $\mathscr{A}$ из предыдущего случая, и покажем, что он и здесь различит рассматриваемую пару слов.		
			\begin{itemize}
				\item Читаем $(xy)^a(yx)^b(xy)^c$
				
				$0.(xy)^a = 0$, так как $c_1 | a$. $0.(yx)^b = s$ по выбору вершины $s$; автомат переходит к циклу $c_2$. $s.(xy)^c = s$, поскольку $c_2 | c$. 
				
				\item Читаем $(yx)^c(xy)^b(yx)^a$
				
				$0.(yx)^c = s$, так как $b \equiv_{c_1} c$; автомат переходит к циклу $c_2$. $s.(xy)^b = v$, где $v$ - вершина цикла длины $c_2$ перестановки $xy$, не совпадающая с $s$, поскольку $c_2 \not | b$. $v.(yx)^a = v$, поскольку $c_1 | a$.
			\end{itemize}
			Так как $v$ и $s$ не совпадают, также приходим к противоречию.
			
			\item Случай $c_1 | c$ и $c_2 | b$ симметричен первому, поэтому можно построить аналогичный автомат. Остальные случаи аналогичны рассмотренным с точностью до смены длин циклов местами.
		\end{enumerate}
	
		Получается, что для любой перестановки, представляемой в виде объединения двух циклов, хотя бы одно из чисел $a$, $b$, $c$ делится на наименьшее общее кратное их длин (то есть, делится на обе длины одновременно).
		
		Предположим теперь, что k-перестановка представима в виде объединения трех циклов с длинами $c_1$, $c_2$, $c_3$. Для каждой пары циклов, как мы уже доказали, наименьшее общее кратное их длин делит хотя бы одно из чисел $a$, $b$, $c$. Всего таких пар в этом случае три. В зависимости их распределениям по числам $a$, $b$, $c$, можно также рассмотреть несколько случаев.
		
		\begin{enumerate}
			\item Все три НОК-а делят один показатель. Тогда этот показатель делится и на $\lcm(c_1, c_2, c_3)$;
			\item Два из трех НОК-ов делят один показатель. Без ограничения общности, $\lcm(c_1, c_2) | a$ и $\lcm(c_1, c_3) | a$. Тогда очевидно $\lcm(c_1, c_2, c_3) | a$;
			\item На каждый из показателей приходится по одной паре. Без ограничения общности, $\lcm(c_1, c_2) | a$ и $\lcm(c_1, c_3) | b$, $\lcm(c_2, c_3) | c$. Поскольку, $c_1 | a$ и $c_1 | b$, $c_1 | (b-a)$, а по Теореме \ref{b is a plus c} $b-a = c$, значит и $c_1 | c$. Получается, что $c$ делится на все три длины циклов, а значит делится на их НОК. 
		\end{enumerate}
		Значит, для порядков, соответствующие которым перестановки разбиваются на три цикла, теорема тоже доказана.
		
		Теперь описанные ранее случаи представим как базу индукции по наименьшему числу "циклов" $n$, на которые можно разбить порядок (то есть соответствующую ему перестановку).
		Предположим, что для всех  $n < m$ теорема доказана. Докажем для $n=m$. Пусть некоторый порядок есть $\lcm(c_1, c_2, ..., c_n)$. По предположению индукции НОК любых $n-1$-их длин циклов делит хотя бы одно из чисел $a$, $b$, $c$. Поскольку таких $n-1$-наборов больше чем 3, хотя бы на одно из этих чисел придется хотя бы 2 набора. Без ограничения общности пусть $\lcm(c_1, c_2, ..., c_{n-1}) | a$ и $\lcm(c_2, c_3, ..., c_n) | a$. Но тогда $a$ делится на $\lcm(c_1, c_2, ..., c_n)$. Теорема доказана.
	\end{proof}

\end{document}