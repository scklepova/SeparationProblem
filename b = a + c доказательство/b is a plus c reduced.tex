\documentclass[12pt]{article}
\usepackage[utf8]{inputenc}
\usepackage{amsthm}
\usepackage{amsmath}
\usepackage{amsfonts}
\usepackage{mathtools}
\usepackage{graphicx}
\usepackage{mathrsfs}
\usepackage{verbatim}

\usepackage[utf8]{inputenc}
\usepackage[english,russian]{babel}
\usepackage{geometry}
%\usepackage{textcase}
\usepackage{multicol}
\usepackage{listings}
\usepackage{extsizes}
\usepackage{indentfirst}
\usepackage{graphicx}
\usepackage{float}
%\usepackage{algorithmicx}
\usepackage{algorithm}
%\usepackage{algpseudocode}
\usepackage{amsthm}

\newtheorem{lemma}{Лемма}
\newtheorem{definition}{Определение}
\newtheorem{theorem}{Теорема}
\newtheorem{corollary}{Следствие}

\DeclareMathOperator{\lcm}{lcm}
\usepackage{extsizes}
\graphicspath{ {images/} }


\newcommand{\image}[2]{
	\begin{figure}[H]
		\center{
			\includegraphics{#1}
		}
		\caption{#2}
	\end{figure}
}

\newcommand{\scalingimage}[3]{
	\begin{figure}[H]
		\center{
			\includegraphics[scale=#2]{#1}
		}
		\caption{#3}
		\label{fig:#1}
	\end{figure}
}

\begin{document}
		$(xy)^a(yx)^b(xy)^c$ = $(yx)^c(xy)^b(yx)^a$ является тождеством в $S_k$, если для каждого $z$ - порядка $k$-перестановки выполняется хотя бы одно из следующих правил:
		\begin{equation} \label{3blocks 1st cond}
		z|a \hspace{10pt} \text{и} \hspace{10pt} z|(b-c)
		\end{equation}
		\begin{equation}
		z|c \hspace{10pt} \text{и} \hspace{10pt} z|(b-a)
		\end{equation}
		\begin{equation} \label{3blocks 3rd cond}
		z|b \hspace{10pt} \text{и} \hspace{10pt} z|(a+c)
		\end{equation}
	
	\begin{definition}
		Циклической перестановкой из $k$ элементов с шагом $s$ будем называть такую циклическую перестановку, в которой элемент с номером $i$ переходит в элемент с номером $i+s \hspace{5pt} (\mod k)$. 
	\end{definition}
	
	Далее будем считать, что элементы перестановки длины $k$ - это числа от $0$ до $k-1$.
	
	\begin{lemma} \label{existence_of_clockwise_and_vv}
		Существуют такие перестановки $x$ и $y$ из $S_k$, что $xy$ - циклическая перестановка с шагом -1, а $yx$ - циклическая перестановка с шагом 1.
	\end{lemma}
	\begin{proof}
		Рассмотрим перестановку $x$: $i \rightarrow -i (\mod k)$ и $y$: $i \rightarrow -i+1 (\mod k)$.
		$$
		x = 
		\begin{pmatrix}
		0&1&2&3&...&k-3&k-2&k-1\\
		0&k-1&k-2&k-3&...&3&2&1
		\end{pmatrix}
		$$
		
		$$
		y = 
		\begin{pmatrix}
		0&1&2&3&...&k-3&k-2&k-1\\
		1&0&k-1&k-2&...&4&3&2
		\end{pmatrix}
		$$
		
		Тогда 
		
		$xy$: $i \xrightarrow y (-i+1) \xrightarrow x (-(-i+1)) == i-1$,
		
		$yx$: $i \xrightarrow x (-i) \xrightarrow y (-(-i)+1) == i+1$
		
	\end{proof}
	
	\begin{lemma} \label{a-b+c is 0 for odd}
		Пусть $(xy)^a(yx)^b(xy)^c$ = $(yx)^c(xy)^b(yx)^a$ - тождество в $S_k$, где $k$ - нечетное. Тогда $a - b + c \equiv 0 \hspace{5pt} (\mod k)$.
	\end{lemma}
	\begin{proof}
		Зафиксируем перестановки $xy$ и $yx$ из Леммы \ref{existence_of_clockwise_and_vv}.
		Рассмотрим перестановочный автомат, в котором переход по символам осуществляется соответствующими перестановками $x$ и $y$. Тогда, чтобы $(xy)^a(yx)^b(xy)^c$ = $(yx)^c(xy)^b(yx)^a$ было тождеством для такого автомата, требуется, чтобы автомат закончил читать обе части равенства в одном состоянии, то есть	
		\begin{equation}
		(-a) + b + (-c) \equiv c + (-b) + a \hspace{5pt} (\mod k)
		\end{equation}
		что эквивалентно 
		\begin{equation}
		2(a - b + c) \equiv 0 \hspace{5pt} (\mod k)
		\end{equation}
		Из того, что $k$ нечетно, следует 
		$$
		(a - b + c) \equiv 0 \hspace{5pt} (\mod k)
		$$.
	\end{proof}

	\begin{lemma} \label{a-b+c is 0 for even}
		Пусть $(xy)^a(yx)^b(xy)^c$ = $(yx)^c(xy)^b(yx)^a$ - тождество в $S_k$, где $k$ - четное. Тогда $a - b + c \equiv 0 \hspace{5pt} (\mod \frac{k}{2})$.
	\end{lemma}
	\begin{proof}
		Рассмотрим перестановочный автомат, как в доказательстве Леммы \ref{a-b+c is 0 for odd}. Аналогично, получим	
		\begin{equation}
		2(a - b + c) \equiv 0 \hspace{5pt} (\mod k)
		\end{equation}
		Из того, что $k$ четно, следует 
		$$
		(a - b + c) \equiv 0 \hspace{5pt} (\mod \frac{k}{2})
		$$.
	\end{proof}

	\begin{corollary} \label{a-b+c is 0 mod lcm/2}
		Пусть $(xy)^a(yx)^b(xy)^c$ = $(yx)^c(xy)^b(yx)^a$ - тождество в $S_k$. Тогда $a - b + c \equiv 0 \hspace{5pt} (\mod \frac{\lcm(k)}{2})$.
	\end{corollary}
	\begin{proof}
		Из Леммы \ref{a-b+c is 0 for odd} $a - b + c \equiv 0$ по модулю наименьшего общего кратного всех нечетных чисел, меньших $k$.
		По Лемме \ref{a-b+c is 0 for even} $a - b + c \equiv 0$ по модулю предмаксимальной степени числа 2, не превосходящей $k$.
		Отсюда следует, что $a - b + c \equiv 0 \hspace{5pt} (\mod \frac{\lcm(k)}{2})$.
	\end{proof}

	\begin{theorem}
		Пусть $(xy)^a(yx)^b(xy)^c$ = $(yx)^c(xy)^b(yx)^a$ - тождество в $S_k$ и $a + b + c \le \frac{\lcm(k)}{2}$. Тогда $b = a+c$.
	\end{theorem}
	\begin{proof}
		Сразу заметим, что ни одно из чисел $a$, $b$, $c$ не равно 0, так как в противном случае мы будем рассматривать тождество другого типа.
		
		Из Следствия \ref{a-b+c is 0 mod lcm/2} вытекает, что \hspace{5pt} 
		$$
			b - a - c = \frac{\lcm(k)}{2}m,
		$$ где $m \in \mathbb{Z}$. 
		Откуда
		$$
			b = a + c + \frac{\lcm(k)}{2}m > 0.
		$$
		Получим цепочку неравенств 
		$$
			0 < a + c + \frac{\lcm(k)}{2}m < \frac{\lcm(k)}{2} + \frac{\lcm(k)}{2}m = \frac{\lcm(k)}{2}(m + 1),
		$$то есть $m > -1$.
		
		С другой стороны, подставим $b$ в $a+b+c$, получим
		$$
			2(a + c) + \frac{\lcm(k)}{2}m \le \frac{\lcm(k)}{2}
		$$ или 
		$$
			2(a + c) \le \frac{\lcm(k)}{2}(1-m)
		$$
		Поскольку сумма $a$ и $c$ должна быть положительным числом, требуется
		$$
			\frac{\lcm(k)}{2}(1-m) > 0,
		$$ то есть $m < 1$.
		
		Значит, при заданных ограничениях $m = 0$, что влечет $b = a + c$.
	\end{proof}

	\begin{corollary}
		Для кратчайшего тождества вида $(xy)^a(yx)^b(xy)^c$ = $(yx)^c(xy)^b(yx)^a$ выполняется $b = a + c$.
	\end{corollary}
	\begin{proof}
		Достаточно показать, что существуют тождества, для которых $a + b + c \le \frac{\lcm(k)}{4}$ (тогда кратчайшее тождество также удовлетворяет этому условию, а по теореме для всех тождеств с таким свойством выполняется $b = a + c$).
		
		Пусть $m = 2k/3$, $a := lcm(m)$, $c := lcm(k - m) \cdot P(m)$, $b := a+c$, где $P(m)$ - произведение всех простых и степеней простых чисел из множества $\{m+1, ..., k\}$. $a$ и $c$ взяты из доказательства длины тождества из двух блоков. Оттуда же понятно, что любой порядок перестановки делит или $a$, или $c$, а, благодаря выбору $b$, делит и соответствующую разность. Длина такого тождества, конечно, в два раза больше длины тождества из двух блоков ($e^{\frac{2}{3}k + O(\frac{k}{\log k})}$), однако все равно асимптотически меньше, чем $\frac{\lcm(k)}{4}$ (который  равен $e^{k + O(\frac{k}{\log k})}$).
	\end{proof}

	Теперь, вооружившись утверждением о связи показателей степеней рассматриваемого тождества, можно доказать обратное утверждение, т.е. если $(xy)^a(yx)^b(xy)^c$ $\equiv_k$ $(yx)^c(xy)^b(yx)^a$, то выполняется хотя бы одно из условий \ref{3blocks 1st cond} - \ref{3blocks 3rd cond}. Для этого нам понадобится доказать еще несколько лемм.
	Однако заметим сразу, что равенство $b = a + c$ делает истинной вторые части утверждений \ref{3blocks 1st cond} - \ref{3blocks 3rd cond}, если первые истинны.
	
	\begin{lemma}
		 Если $(xy)^a(yx)^b(xy)^c$ $\equiv_k$ $(yx)^c(xy)^b(yx)^a$, тогда хотя бы одно из чисел $a$, $b$, $c$ делится на 2.
	\end{lemma}
	\begin{proof}
		От противного. Пусть $a$, $b$, $c$ нечетны. Тогда $b \ne a+c$, поскольку их четность не совпадает. Противоречие.
	\end{proof}

	\begin{lemma}
		Если $(xy)^a(yx)^b(xy)^c$ $\equiv_k$ $(yx)^c(xy)^b(yx)^a$ и $k \ge 4$, тогда хотя бы одно из чисел $a$, $b$, $c$ делится на 3.
	\end{lemma}
	\begin{proof}
		От противного. Пусть ни одно из чисел $a$, $b$, $c$ не делится на 3. Тогда возможны лишь два варианта:
		\begin{enumerate}
			\item $a \equiv_3 c \equiv_3 1$ и $b \equiv_3 2$
			\item $a \equiv_3 c \equiv_3 2$ и $b \equiv_3 1$
		\end{enumerate}
	
		(В остальных случаях хотя бы одно из чисел оказывается кратным трем)
		
		Рассмотрим автомат относительно символов $xy$, $yx$ на рисунке \ref{fig:3division}. Такой автомат можно получить, взяв за перестановку по $x$ $(0)(1,2,3)$, по $y$ - $(1)(3, 2, 0)$. Начальное состояние 0.
		
		В первом случае автомат закончит читать левую часть тождества в состоянии 3 (после прочтения $(xy)^a$ окажется в состоянии 1, затем, прочитав $(yx)^b$ придет в состояние 3 в цикле), а правую - в состоянии 1 (после прочтения $(yx)^c$ окажется в состоянии 3 и в нем останется после $(xy)^b$).
		
		\scalingimage{3division}{0.5}{}
		
		Во втором случае автомат закончит читать левую часть тождества в состоянии 1 (после прочтения $(xy)^a$ окажется в состоянии 2, затем, прочитав $(yx)^b$ останется в состоянии 2), а правую - в состоянии 2 (после прочтения $(yx)^c$ окажется в состоянии 1 и перейдет в состояние 2 после $(xy)^b$).
		
		Получили противоречие с тем, что данная пара - тождество.
	\end{proof}

	\begin{lemma}
		Если $(xy)^a(yx)^b(xy)^c$ $\equiv_k$ $(yx)^c(xy)^b(yx)^a$ и $k \ge 4$, тогда хотя бы одно из чисел $a$, $b$, $c$ делится на $k$.
	\end{lemma}
	\begin{proof}
		От противного. Пусть ни одно из чисел $a$, $b$, $c$ не делится на $k$. Разберем несколько случаев.
		\begin{enumerate}
			\item $b+c \not \equiv_k 0$
		\end{enumerate}
	\end{proof}

\end{document}