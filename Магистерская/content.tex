\sectionwithoutnumber{ВВЕДЕНИЕ}

Задача о различении двух наборов входных данных - одна из самых простых вычислительных задач, которые можно себе представить. Обычно, входные данные представлены в виде двух строк $u$ и $v$ над конечным алфавитом $\Sigma$ и известны заранее. 

В мощной вычислительной модели, такой как RAM, задача решается за константную память (помимо памяти, занимаемой строками): нам достаточно одного регистра для того, чтобы найти позицию, в которой различаются $u$ и $v$. 

Однако для более слабой модели, например, для конечных автоматов, задача существенно усложняется, и ее уже нельзя решить за константую память. Задача об определении минимального размера конечного автомата, различающего два данных слова, NP-трудна. Более того, не всё известно об асимптотике минимального количества вершин, необходимого, чтобы построить различающий автомат для каждой пары слов, длина которых меньше либо равна заданному наперед числу. Это и есть задача о различении слов автоматами, и на данный момент известны только верхняя и нижняя границы искомой функции, между которыми довольно большой разрыв.

Далее в работе будут даны все необходимые определения, задача будет поставлена более формально, также будет показана эквивалентность этой задачи поиску тождеств в полгруппе преобразований и группе перестановок. Более того, будут приведены серии тождеств, одна из которых позволяет слегка поднять нижнюю границу искомой функции. 

\newpage
\section{Основные определения и постановка задачи}

\subsection{Основные определения}
Пусть $\mathscr{A} = (\Sigma, Q, \delta, s, T)$ – детерминированный конечный автомат, где 
\\ $\Sigma$ – входной алфавит, из которого формируются слова, принимаемые автоматом, $\Sigma \ne \emptyset$,
\\ Q – множество состояний автомата, $Q \ne \emptyset$,
\\ $\delta$ – функция переходов, определенная как отображение $\delta: Q \times \Sigma \rightarrow Q$, 
\\ s – начальное состояние, $s \in Q$
\\ T – множество терминальных (конечных) состояний, $T \subseteq Q$.

\begin{definition}
	Автомат принимает слово, если по окончании его обработки он находится в терминальном состоянии.
\end{definition}
\begin{definition}
	Пусть $u$ и $v$ – слова над алфавитом $\Sigma$. Говорят, что автомат различает слова $u$ и $v$, если он принимает одно из них и не принимает другое. 
\end{definition}

\begin{example}
	Автомат на рисунке \ref{automaton_example} различает слова 0010 и 1000.
\end{example}

\image{example.png}{}\label{automaton_example}

\begin{definition}
	Пару слов $(u, v)$ будем называть тождеством для некоторого автомата, если он либо принимает и $u$, и $v$, либо отвергает и $u$, и $v$, то есть не различает их.
\end{definition}

Однако, в дальнейшем нам будет удобнее будет считать терминальными все состояния у рассматриваемых автоматов, и в связи с этим подкорректировать определения.

\begin{definition}
	Автомат различает слова $u$ и $v$, если он заканчивает их чтение в разных состояниях, то есть $s.u \ne s.v$. 
\end{definition}

Поскольку вся работа посвящена перестановочным автоматам, необходимо дать и это ключевое определение.
\begin{definition} 
	Автомат $\mathscr{A}$ – перестановочный, если переход из любого состояния по любому символу является перестановкой состояний, или, что тоже самое, для любого символа $x$ из $\Sigma$ и любых состояний $q$ и $p$ $\delta(q, x) \ne \delta(p, x)$.
\end{definition}

\subsection{Постановка задачи}
 
Обозначим за $sep(u, v)$ количество состояний в минимальном детерминированном конечном автомате, различающем $u$ и $v$, за $sep_p(u, v)$ – количество состояний в перестановочном автомате, различающем $u$ и $v$.

\begin{example}
	Можно проверить, что ни один автомат из двух состояний не сможет различить слова 0010 и 1000, а как мы видели из рисунка \ref{automaton_example}, существует автомат с тремя состояниями, различающий эту пару. Значит, $sep(0010, 1000) = 3$.
\end{example}

Обозначим
$$
	S(n) = \max_{u \ne v; |u|, |v| \le n} sep(u, v)
$$ 
и
$$
	S_p(n) = \max_{u \ne v; |u|, |v| \le n} sep_p(u, v)
$$
Задача о различении слов автоматами, известная как Separating Words Problem, состоит в том, чтобы найти хорошую асимптотическую оценку функций $S(n)$ и $S_p(n)$, то есть оценить, сколько состояний должно быть в автомате, различающем две строки длины $n$.

Эту задачу сформулировали Павел Горальчик и Вацлав Коубек в 1986 году \cite{on discerning, remarks on separating}. Они же доказали, что $S(n) = o(n)$. Позже этой задачей занимался Джон Робсон, который в 1989 году \cite{robson 1989} доказал, что $S(n) = O(n^{2/5}(\log n)^{3/5})$ для произвольных автоматов, и в 1996 году \cite{robson 1996} опубликовал статью, в которой было доказано, что $S_p(n) = O(n^{1/2})$.

\subsection{Сведение к поиску тождеств в полугруппах и группах}

Обозначим за $T_k$ полугруппу всех отображений множества ${1, ..., k}$ в самого себя относительно операции композиции отображений. $T_k$ называют полугруппой преобразований на $k$ элементах.
\begin{definition}
	Тождеством в полугруппе $T$ называют пару слов $(u, v)$ такую, что образы $u$ и $v$ под действием любого отображения $\Sigma \rightarrow T$ совпадают как элементы $T$. Длина тождества $(u, v)$ - максимум длин $u$ и $v$. Факт тождественности $u$ и $v$ в $T_k$ будем обозначать как $u \equiv_k v$.
\end{definition}

\begin{definition}
	Полугруппа преобразований детерминированного конечного автомата $\mathscr{A}$ - это подполугруппа $T_{|Q|}$, состоящая из всех отображений $w: q \rightarrow q.w$, где $q \in \Sigma^*$.
\end{definition}

Следующий факт соединяет между собой тождества в полугруппах и разделение слов автоматами.

\begin{fact} \label{identity separation equivalency}
	Для любой пары слов $u$, $v$, $u \equiv_k v$ $ \iff $ $S(u, v) > k$
\end{fact}

Действительно, если $u \equiv_k v$, то это тождество выполняется и для полугруппы преобразований любого конечного детерминированного автомата с $k$ состояниями, откуда получаем $q.u = q.v$ для любого состояния $q$. С другой стороны, если для некоторого отображения $\rho: \Sigma \rightarrow T_k$ $\rho(u) \ne \rho(v)$ в $T_k$, тогда преобразования $\rho(a), a \in \Sigma$ могут быть использованы для создания автомата с $k$ состояниями, отличающего $u$ от $v$.

Известно, что задача о проверке $u \equiv_k v$ принадлежит классу coNP-complete для любого $k > 2$. Поэтому исходя из Факта \ref{identity separation equivalency}, задача о проверке $S(u, v) \le k$ является NP-полной. К тому же, задача о различении слов автоматами эквивалентна поиску асимптотики минимальной длины тождества в $T_k$.

До недавнего времени самым коротким тождеством в $T_k$ было 
\begin{equation} \label{unary identity}
	x^{k-1} = x^{k-1+lcm(k)},
\end{equation} 
где $lcm(k)$ - это наименьшее общее кратное всех чисел от 1 по $k$. Однако недавно было найдено новое тождество, которое короче \ref{unary identity}, если $k$ - простое или степень простого числа \cite{lower bounds}.

Имеет место аналогичная Факту \ref{identity separation equivalency} связь различения слов перестановочными автоматами и тождеств в группе перестановок $S_k$.

\begin{fact}
	Для любой пары слов $u$, $v$, $u \equiv_k v$ в $S_k$ $ \iff $ $S_p(u, v) > k$
\end{fact}

Самое короткое тождество в $S_k$, дающее почву для нижней оценки функции $S_p(k)$ имеет вид 
$$
	x^{lcm(k)} = 1
$$
Было несколько попыток найти более короткие групповые тождества. Существование тождества длины $O(e^{\sqrt[]{n \log n}})$ было доказано в статье \cite{short group identity 1}; основная идея заключается в применении функции Ландау о максимальном порядке перестановки. Совсем недавно было также доказано существование тождества длины $O(e^{\log^4 n \log \log n})$ \cite{short group identity diameter of Caleys graph}, основанного на новых результатах о диаметре графа Кэли $S_k$. В данной работе будет представлена серия тождеств, показывающая, что $S_p(n) \ge \frac{3}{2}\log n + o(\log n)$.

\newpage
\subsection{Полезные факты}
Вставить сюда ограничения: 
	одинаковая длина слов,
	бинарный алфавит,
	начинаются и заканчиваются по-разному
	над алфавитом (xy) (yx) 

\newpage
\section{Серии тождеств}
Известно, что группа перестановок $S_k$ удовлетворяет тождеству \ref{unary identity}  и его бинарному собрату $x^{lcm(k)} \equiv_k y^{lcm(k)}$.


\subsection{Тождетсва из двух блоков}

\subsection{Тождества из трех блоков}
%\include{3blocks}

\subsection{Тождества из четырех и более блоков}

\newpage
\section{Заключение}