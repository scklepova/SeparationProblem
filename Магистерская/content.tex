\sectionwithoutnumber{ВВЕДЕНИЕ}

Задача о различении двух наборов входных данных - одна из самых простых вычислительных задач, которые можно себе представить. Обычно, входные данные представлены в виде двух строк $u$ и $v$ над конечным алфавитом $\Sigma$ и известны заранее. 

В мощной вычислительной модели, такой как RAM, задача решается за константную память (помимо памяти, занимаемой строками): нам достаточно одного регистра для того, чтобы найти позицию, в которой различаются $u$ и $v$. 

Однако для более слабой модели, например, для конечных автоматов, задача существенно усложняется, и ее уже нельзя решить за константую память. Задача об определении минимального размера конечного автомата, различающего два данных слова, NP-трудна. Более того, не всё известно об асимптотике минимального количества вершин, необходимого, чтобы построить различающий автомат для каждой пары слов, длина которых меньше либо равна заданному наперед числу. Это и есть задача о различении слов автоматами, и на данный момент известны только верхняя и нижняя границы искомой функции, между которыми довольно большой разрыв.

Далее в работе будут даны все необходимые определения, задача будет поставлена более формально, также будет показана эквивалентность этой задачи поиску тождеств в полгруппе преобразований и группе перестановок. Более того, будут приведены серии тождеств, одна из которых позволяет слегка поднять нижнюю границу искомой функции. 

\newpage
\section{Основные определения}
Пусть $\mathscr{A} = (\Sigma, Q, \delta, s, T)$ – детерминированный конечный автомат, где 
\\ $\Sigma$ – входной алфавит, из которого формируются слова, принимаемые автоматом, $\Sigma \ne \emptyset$,
\\ Q – множество состояний автомата, $Q \ne \emptyset$,
\\ $\delta$ – функция переходов, определенная как отображение $\delta: Q \times \Sigma \rightarrow Q$, 
\\ s – начальное состояние, $s \in Q$
\\ T – множество терминальных (конечных) состояний, $T \subseteq Q$.

\begin{definition} 
	Автомат $\mathscr{A}$ – перестановочный, если переход из любого состояния по любому символу является перестановкой состояний, или, что тоже самое, для любого символа $x$ из $\Sigma$ и любых состояний $q$ и $p$ $\delta(q, x) \ne \delta(p, x)$.
\end{definition}
\begin{definition}
	Автомат принимает слово, если по окончании его обработки он находится в терминальном состоянии.
\end{definition}
\begin{definition}
	Пусть $u$ и $v$ – слова над алфавитом $\Sigma$. Говорят, что автомат различает слова $u$ и $v$, если он принимает одно из них и не принимает другое. 
\end{definition}
\begin{definition}
	Пару слов $(u, v)$ будем называть тождеством для некоторого автомата, если он либо принимает и $u$, и $v$, либо отвергает и $u$, и $v$, то есть не различает их.
\end{definition}

Обозначим за $sep(u, v)$ количество состояний в минимальном детерминированном конечном автомате, различающем $u$ и $v$, за $sepp(u, v)$ – количество состояний в перестановочном автомате, различающем $u$ и $v$. 
Например, автомат на рисунке 1 различает слова 0010 и 1000.



\newpage
\section{Постановка задачи}

\newpage
\section{Полезные факты}

\newpage
\section{Серии тождеств}

\subsection{Тождетсва из двух блоков}

\subsection{Тождества из трех блоков}

\subsection{Тождества из четырех и более блоков}

\newpage
\section{Заключение}